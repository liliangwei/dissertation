\begin{abstract}
In this paper, we present a novel parallel computing method on solving linear elasticity problems on unstructured meshes efficiently. The numerical method of our parallel computing scheme is based on the weak Galerkin (WG) finite element method which has been recently developed by Dr. Junping Wang and Dr. Xiu Ye. The weak Galerkin finite element method refers to a general finite element method on tackling a variety of partial differential equations. The main feature is that the differential operators are discretized and replaced by weak operators. Through weak functions, a connection between weak operators and unknown variables is reconstructed by solving the elemental matrices for each element. The key idea of utilizing WG method for solving linear elasticity equation is by converting the weak strain and stress tensors to the weak operators from employing the concept of discrete weak gradients. 

%Linear elasticity is the equation describes how solid objects stress and strain distribute due to the external and internal prescribed loading conditions. This equation requires the materials as continua. 
A long-range goal of this research is to solve large-scale fluid structure interaction problems on parallel computers. For the purpose of solving large-scale fluid structure interaction problems, we will adopt a partitioned approach. In this approach, the governing equations for fluid and the displacement of the structure are calculated by two different solvers. Since an in-house fluid solver for Naiver-Stokes equations has been well-established, now we present an accurate and efficient solid solver which is capable on solving linear elasticity in parallel fashion.

To achieve high efficient computational process for structural dynamics, the non-overlapping domain decomposition is introduced to empower the WG method. We present two different approaches to implement the parallel computing. Initially, we combine the classic continuous Galerkin (CG) finite element with weak Galerkin (WG) finite element method together and result in a hybrid element. The hybrid element inherits the discontinuous feature from WG method and the computational efficiency from CG method. After the Schur complement method, the accuracy and scalability have been tested 
The more advanced approach is to implement the duality concept to split the computational space into primal and dual spaces. The connection of adjacent subdomains is implemented through the balancing domain decomposition with constraints (BDDC) which is originally proposed by Mandel\cite{mandel1993balancing}. Locally over each subdomain, matrices are constructed for interior and interface quantities separately. A powerful preconditioner for interface problem is constructed in the parallel fashion through a global primal space and a local dual space. Therefore, such interface related quantities are passed over to their adjacent subdomains through inter-processor communication library (MPI). After the convergence of interface problem, the rest unknown variables are recovered locally. 

This dissertation is structured as follow:

In Chapter 1, we introduce the background and objective of this dissertation. We provide the preliminaries which are necessary for the following presentation. We derive the bilinear form of second order elliptical equation and linear elasticity equation. 

In Chapter 2, we discuss the weak Galerkin finite element method and the bilinear form of linear elasticity equation.The WG finite element method is based on the variational form of equations. It is compatible with general polygons for finite element computational domain. The stiffness matrix derived from the WG method is symmetric and positive definite. Due to the flexibility of the polynomials basis functions, it's convenient to obtain high order accurate solutions. The convergence rate for the WG method is bounded by the lowest order.

In Chapter 3, we design a novel parallel computing method to solve linear elasticity equation. The key idea of the WG method for solving linear elasticity equation is to replace its gradients by discrete weak strain and stress operators. We develop a novel hybrid element which combines the elements of both weak Galerkin (WG) finite element method and continuous Galerkin (CG) finite element method. The new hybrid element inherits the discontinuous feature of the WG method. In every hybrid element, we can insert an arbitrary number of CG elements in one single WG element. The hybrid element provides a second order of accuracy for both linear and nonlinear elastic equation. Meanwhile, the superlinear speedup is obtained.

In Chapter 4, we develop a novel parallel computing method to efficiently solve linear elasticity problems on unstructured meshes. To enable parallel computation, the computational grid is divided into arbitrary number of subdomains. The connection of adjacent subdomains is realized through the balancing domain decomposition with constraints (BDDC) which was originally proposed in \cite{mandel1993balancing}. Locally over each subdomain, matrices are constructed for interior and interface quantities separately. Consequently, interface related matrices are passed over to their adjacent subdomains through MPI libraries. MPI communications are employed to construct a smaller global preconditioner and distribute most computational effort to local processor. The novel WG-BDDC parallel algorithm achieves outstanding scalability up to 600 processors. Our numerical results also demonstrate that the WG-BDDC method obtained optimal order of accuracy for both 2nd-order and 3rd-order spatial discretization schemes. Moreover, the condition numbers for all test problems are well bounded.

In Chapter 5, We conclude the current stage and explore the future potential work.

Appendix A, we present computational fluid dynamics (CFD) simulations of the blood flow in stenotic peripheral arteries with idealized geometries. These arteries are typically simplified as axisymmetric constriction in straight tubes. The hydrodynamics of blood flow in these arteries are modeled using unsteady incompressible Navier-Stokes equations. The 3D computational domain is represented by unstructured meshes with all hexahedral elements. An efficient pressure-based Finite Volume Method(FVM)\cite{liang2007large} was implemented to solve these equations. Our simulations include computational geometries with a wide range of narrowing degrees of stenoses, from 40\% to 80\% luminal area reduction. The number of stenoses ranges from 1 to 7. Several different spatial intervals were considered between adjacent stenoses. The upstream flow condition was implemented by using measured peripheral artery flow which at the Reynolds number of 500.


\textbf{Keyword} : weak Galerkin, finite element method, parallel computing, linear elasticity, message passing interface, continuous Galerkin, domain decomposition, balancing domain decomposition by constraints, polygonal meshes

\end{abstract}
