\begin{abstract}
In this paper, we present a novel parallel computing method to efficiently solve linear elasticity problems on unstructured meshes. The implementation of our parallel method is based on the weak Galerkin (WG) finite element method which has been recently developed by Wang and Ye and Mu et al. The core idea of the WG method for solving linear elasticity equation is to introduce weak strain and stress tensors by using the concept of discrete weak gradients. The weak Galerkin finite element method refers to a general finite element method to solve partial differential equations. The main feature is that the differential operators are calculated through the weak functions and reconstructed by solving the relatively inexpensive elemental matrices on each element.

Linear elasticity is the equation describe how solid objects stress and strain distribute due to the external and internal prescribed loading condition. This equation requires the materials as continua. On the purpose of solving large-scale fluid structure interaction problems, the partitioned approach, the governing equation for fluid and displacement of the structure are calculated in two different solver. We present an accurate and efficient solid solver which is compatible for high order fluid solvers for Navier-Stokes equations.

To enable parallel computation, the computational grid is split into arbitrary number of subdomains.We present two different approaches to implement the parallel computing. Firstly, we combine the classic continuous Galerkin (CG) finite element with weak Galerkin finite element method together and develop a hybrid element. The hybrid element inherit the discontinuous feature from WG method and the computational efficiency from CG method. 
The other method is to implement the primal and dual spaces to split the computational spaces for ech subdomain. The connection of adjacent subdomains is realized through the balancing domain decomposition with constraints (BDDC) which was originally proposed in. Locally over each subdomain, matrices are constructed for interior and interface quantities separately. Subsequently, interface related ma- trices are passed over to their adjacent subdomains through MPI. 

In Chapter 1, we introduce the background and meaning for this thesis. We provide the preliminaries which are necessary for the following presentation. We derived the bilinear form of second order elliptical equation and linear elasticity equation. 

In Chapter 2, we discuss the weak Galerkin finite elemet method and the bilinear form of linear elasticity equation.The WG finite element method is based on the variational form of equations. It is compatible for general polygons on a finite element computational domain. The computational matrix derived from WG method is symmetric and positive definite. Due to the flexibility of the polynomials basis functions, it's convenient to obtain high order accuracy solutions. The convergence rate for WG method is bounded by the lowest order.

In Chapter 3, we design a novel parallel computing method to efficiently solve elastic equation. The core idea of the WG method for solving linear elastic equation is to replace its gradients after the integration by parts by discrete weak strain and stress tensors. We develop a novel hybrid element which combines the elements of both weak Galerkin (WG) finite element method and continuous Galerkin (CG) finite element method. The new hybrid element inherits the discontinuous feature of the WG method. We insert an arbitrary number of CG elements in one single WG element. The hybrid element provides a second order of accuracy for both linear and nonlinear elastic equation. The superlinear speedup is observed.

In Chapter 4, we develop a novel parallel computing method to efficiently solve linear elasticity problems on unstructured meshes. The core idea of the WG method for solving linear elasticity equation is to introduce weak strain and stress tensors by using the concept of discrete weak gradients. To enable parallel computation, the computational grid is split into arbitrary number of subdomains. The connection of adjacent subdomains is realized through the balancing domain decomposition with constraints (BDDC) which was originally proposed in \cite{mandel1993balancing}. Locally over each subdomain, matrices are constructed for interior and interface quantities separately. Subsequently, interface related matrices are passed over to their adjacent subdomains through MPI. MPI communications are used to help construct a smaller global matrix leaving most of computational operations locally to each processor. The designed WG-BDDC parallel algorithm achieves outstanding scalability by testing over 600 processors. Our numerical results also demonstrate that the WG-BDDC method pos- sesses designed orders of accuracy for both 2nd-order and 3rd-order spatial discretization schemes. Moreover, condition numbers for all test problems are well bounded demonstrating the stability of WG-BDDC method for parallel processing.

In Chapter 5, We conclude the current stage and explore the future potential work.

\textbf{Keyword} : weak Galerkin, finite element method, parallel computing, linear elasticity, message passing interface, continuous Galerkin, domain decomposition, balancing domain decomposition by constraints, polygonal meshes

\end{abstract}
