\begin{abstract}
In this dissertation, we present a novel parallel computing method for efficiently solving elasticity equations on unstructured meshes. The numerical method of our parallel computing scheme is based on the weak Galerkin (WG) finite element method which is newly developed by Dr. Junping Wang and Dr. Xiu Ye. The WG finite element method refers to a general finite element method on tackling a variety of partial differential equations. The distinct feature of the WG method is that the differential operators are discretized and replaced by the weak operators. Through the weak functions, a connection between the weak operators and unknown variables is connected by solving the elemental matrices for each element. Utilizing the WG method for solving the elasticity equation is converting the weak strain and stress tensors to the weak operators by employing the concept of discrete weak gradients and weak divergences. 

%Linear elasticity is the equation describes how solid objects stress and strain distribute due to the external and internal prescribed loading conditions. This equation requires the materials as continua. 
A long-term goal of this research is to solve large-scale fluid-structure interaction problems on parallel computers. On the purpose of solving large-scale fluid-structure interaction problems, we choose to employ a partitioned approach and coupled by the Implicit-Explicit coupling method. In this approach, the governing equations of the fluid and structural parts are calculated by two individual solvers. Since an in-house fluid solver for solving the Naiver-Stokes equations has been developed and published, an accurate and efficient solid solver, which is capable of solving elasticity equation on parallel computers, is needed now.

To achieve a highly efficient computational process dealing with the structural dynamics, a non-overlapping domain decomposition scheme is introduced to parallelize the WG solver. We present two different approaches to implement the parallel computing. Initially, we combine the classic continuous Galerkin (CG) finite element method with the WG finite element method  and design a hybrid WG-CG element. The hybrid element inherits the discontinuous feature from the WG method and the computational efficiency from the CG method. After implementing the Schur complement method, the accuracy and scalability have been demonstrated. 
A more advanced approach for 2 dimensional problem is to implement the duality concept to split the computational space into primal and dual spaces. The connection of adjacent subdomains is implemented through the balancing domain decomposition with constraints (BDDC) method which is originally proposed by Mandel\cite{mandel1993balancing}. Locally over each subdomain, matrices are constructed for interior and interface quantities separately. An efficient preconditioner for interface problem is constructed in the parallel fashion through a global primal space and a local dual space. Therefore, such interface related quantities are passed over to their adjacent subdomains through inter-processor communication library, Message Passing Interface (MPI). After the convergence of the interface problem, the remaining local unknown variables could be recovered locally. 

This dissertation is structured as follow:

In Chapter 1, we introduce the background and objectives of this dissertation. We provide the preliminaries which are important and necessary for the following discussions. We derive the bilinear form of second order elliptical equation and the elasticity equation. 

In Chapter 2, we discuss the weak Galerkin finite element method and the bilinear form of linear elasticity equation.The WG finite element method is based on the variational form of the governing equations. Different types of polygons can be applied as finite element for the WG method. The stiffness matrix derived from the WG method is symmetric and positive definite. Due to the flexibility of the polynomials basis functions, it is simple to construct high order element. The order of accuracy for the WG method is determined by the highest order of basis function in the element.

In Chapter 3, we design a novel parallel computing method for solving the elasticity equation. We develop a novel hybrid WG-CG element which combines the elements of weak Galerkin (WG) finite element method and classic continuous Galerkin (CG) finite element method. The new hybrid element inherits the discontinuous feature of the WG method. In the hybrid element, we can insert multiple CG elements in one WG element. Each hybrid element is treated as a computational subdomain calculated by individual processor.  In our numerical examples, second order of accuracy is demonstrated for interior and boundary unknown variables. A superlinear speedup is obtaind with the scalability up to 60 processors.

In Chapter 4, we develop a novel parallel solver for the linear elasticity problems on unstructured meshes. To enable parallel computation, the computational domain is partitioned into multiple subdomains. The connection of adjacent subdomains is implemented through the balancing domain decomposition with constraints (BDDC) method which was originally proposed by Mandel \cite{mandel1993balancing}. Locally, over each subdomain, the matrices are constructed for interior and interface quantities concurrently. Consequently, the interface related matrices are computed by their adjacent subdomains through MPI libraries. MPI communications are employed to construct an efficient global preconditioner and distribute the major computational cost to local processors. The novel WG-BDDC parallel algorithm achieves superlinear speedup up to 600 processors. The highest speedup ratio achieved at $ 123\% $. Our numerical results demonstrate that the WG-BDDC method obtained optimal order of accuracy for both 2nd-order and 3rd-order spatial discretization schemes. In addition, the condition numbers of the Lanczos matrix from PCG iteration for all test problems are well bounded.

In Chapter 5, we conclude the dissertation and outlook the future work.

%In Appendix A, we present computational fluid dynamics (CFD) simulations of the blood flow in stenotic peripheral arteries with idealized geometries. These arteries are typically simplified as axisymmetric constrictions in straight vessels. The hydrodynamics of blood flow in these arteries are modeled using unsteady incompressible Navier-Stokes equations. The 3D computational domain is represented by unstructured meshes with all hexahedral elements. An efficient pressure-based Finite Volume Method(FVM)\cite{liang2007large} was implemented to solve these equations. Our simulations include computational geometries with a wide range of narrowing degrees of stenoses, from 40\% to 80\% luminal area reduction. The number of stenoses ranges from 1 to 7. Several different spatial intervals were considered between adjacent stenoses. The upstream flow condition was implemented by using measured peripheral artery flow at a Reynolds number of 500.


\textbf{Keywords} : weak Galerkin, finite element method, parallel computing, linear elasticity, message passing interface, continuous Galerkin, domain decomposition, balancing domain decomposition by constraints, polygonal meshes

\end{abstract}
