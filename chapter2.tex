\chapter{Numerical Method}
\label{chapter2}

%--------------------------------------
\section{Mathematical Models and Numerical Method}
\label{sec:equations}

\subsection{Linear Elastic Equation and Weak Galerkin Method}
Consider an elastic body subject to an exterior force $ \mathbf{f} $, we denote the computational domain as $ \Omega $ and its continuous boundary as $ \Gamma = \partial \Omega $. The governing equation for linear elasticity can be written as

\begin{equation}
\nabla \cdot \sigma(\mathbf{u}) = \mathbf{f}, \qquad in \quad \Omega 
\end{equation}
\begin{equation}
\mathbf{u} = \hat{\mathbf{u}}, \qquad on \quad \Gamma
\end{equation}
where $ \sigma(\mathbf{u}) $ is the symmetric Cauchy stress tensor. For linear, isotropic and homogeneous materials, the stress-strain relation is
\begin{equation}
\sigma(\mathbf{u}) = 2 \mu \varepsilon(\mathbf{u}) + \lambda (\nabla \cdot \mathbf{u}) \mathbf{I}
\end{equation}
where $ \varepsilon(\mathbf{u}) = \frac{1}{2} (\nabla \mathbf{u} + \nabla \mathbf{u}^{T}) $, $ \mu $ and $ \lambda  $ are Lame indices which can be written as 
\begin{equation}
\lambda = \frac{E\mu }{(1 + \mu) (1 - 2\mu)}
\end{equation}
\begin{equation}
\mu = \frac{E}{2(1+\mu)}
\end{equation}
where $ E $ is the elasticity modulus and $ \mu $ is the Poisson's ratio.



The weak function on the domain is $ \mathbf{u} = \{\mathbf{u}_0, \mathbf{u}_b \}, \quad \mathbf{u}_0 \in L^{2} (T) $. The first function $ \mathbf{u}_0 $  represents the interior domain of the function $ \mathbf{u} $ . The second function $ \mathbf{u}_b $ represents the value of function $ \mathbf{u} $ on the boundary of domain $ T $ . The key notion is that for two function $ \mathbf{u}_0 $  and $ \mathbf{u}_b $  are independent with each other along . The weak function is defined as
\begin{equation}
V_{h} = \{ \mathbf{v} = \{ \mathbf{v}_0, \mathbf{v}_b \} : \mathbf{v}_0 \in P_{j} (T^0), \mathbf{v}_b \in P_{l}(e), e \subset \partial T\}
\end{equation}

The key of the weak Galerkin method is to approximate the solution in the weak discrete space $ S(T) $. The discrete weak gradient $ \nabla_{w} \mathbf{u} \in [P_{r} (T)]^{d} $ for $ \mathbf{u} \in V_{h} $ on each element $ T $:
\begin{equation}
(\nabla_w \mathbf{u}, \mathbf{q})_{T} = - (\mathbf{u}_0, \nabla \cdot \mathbf{q})_{T} + \langle \mathbf{u}_b, q \cdot \mathbf{n} \rangle_{\partial T}
\end{equation}

For the discrete weak divergence, $ \nabla_{w} \cdot \mathbf{u} \in [P_{r} (T)]^{d} $ is defined
\begin{equation}
(\nabla_w \cdot \mathbf{u}, \mathbf{q})_{T} = - (\mathbf{u}_0, \nabla  \mathbf{q})_{T} + \langle \mathbf{u}_b \cdot \mathbf{n}, q  \rangle_{\partial T}
\end{equation}

Then we can define the weak strain tensor by using the weak gradient
\begin{equation}
\varepsilon_{w} (\mathbf{u})  = \frac{1}{2} (\nabla_w  \mathbf{u} + \nabla_w  \mathbf{u}^T)
\end{equation}

Analogously, we can define the weak stress tensor as
\begin{equation}
\sigma_{w} ( \mathbf{u}) = 2 \mu \varepsilon_{w} ( \mathbf{u}) + \lambda (\nabla_w \cdot  \mathbf{u}) \mathbf{I}
\end{equation}

The bilinear form of governing equation of continuous Galerkin method is following
\begin{equation}
a( \mathbf{u},  \mathbf{v}) = ( \mathbf{f},  \mathbf{v})
\end{equation}

The above approximation function $  \mathbf{u} $ and gradient $ \nabla \mathbf{u} $ is not well defined for the discontinuous feature of weak Galerkin method, the new form is 
\begin{equation}
a( \mathbf{u}_w,  \mathbf{v}_w) + s( \mathbf{u},  \mathbf{v}) = ( \mathbf{f},  \mathbf{u})
\end{equation}
the term $ s( \mathbf{u},  \mathbf{v}) $ is a stabilizer enforcing a weak continuity which measures the discontinuity of the finite element solution. The governing equation in weak form can be introduced by two bilinear equations
\begin{equation}
s( \mathbf{u},  \mathbf{u}) = \sum_{T\in \Omega}^{N} h_{T}^{-1} \langle Q_{b}  \mathbf{u}_0 -  \mathbf{u}_b, Q_b  \mathbf{v}_0 -  \mathbf{v}_b \rangle_{\partial T}
\end{equation}
where $ Q_b $ is the projection from the interior unknown variables to boundary unknown variables. Commonly it is taken as $ 1 $. The discrete bilinear equation has the assemblage form
\begin{equation}
a( \mathbf{u}_w,  \mathbf{u}_w) = \sum_{T \in \Omega}^{N} 2(\mu \varepsilon_w ( \mathbf{u}), \varepsilon_w ( \mathbf{v}))_T + \sum_{T \in \Omega}^{N}(\lambda \nabla \cdot \mathbf{u}, \nabla_w \cdot \mathbf{v})_T
\end{equation}


%--------------------------------------
\section{Existing Numerical Methods Review}
In this section, we present and analyze several most widely used numerical method to solve finite element problems. The details of each method are presented in the following subsections. 
\subsection{Classic Continuous Galerkin Finite Element Method}
Back to 1950s and 1960s, finite element method is arised to solve complex elasticity and structural analysis engineering problems in mechanical and aeronautical field. A. Hrennikoff\cite{hrennikoff1941solution}, R. Courant\cite{courant1994variational} and K.Feng are the earliest pioneers who established this subject. FEM is then proposed as a systematic numerical method to solve variety of partial differential equations. The core characteristic of FEM is that it employs mesh discretization to divide a continuous computational domain so that a big problem is then converted to a set of discrete small problems. That is the source of finite element. Each element represents a small piece of computational sub-domain.

Finite element method is an efficient solution to solve partial differential equations. The core idea is to convert the original partial differential equation to a equivalent bilinear form weak function. Then we partition the computational domain into polygon meshes. In each mesh element, we construct the finite element space. Then the bilinear form is discretized into a summation of finite elemental spaces. The solution is approximated based on the calculation of assembled matrix. More details can be found in \cite{zienkiewicz1977finite, ciarlet2002finite, hughes2012finite, reddy1993introduction}

The variational formulation which generated by the governing equation determines the characteristic of the finite element method. To obtain the variational form, many mathematicians derived several different paths such as Galerkin method, the discontinuous Galerkin method, mixed method, etc. In this chapter, we introduce two most popular method, DG and MFEM. The WG method is inspired from these two method and shares many similarities with them. 

\subsection{Discontinuous Galerkin Finite Element Method}

We shall go through a simple example to review the common features of FEM and the new feature of DG-FEM.

We have a 1-dimensional convection equation with domain $ [0, \pi] $

\begin{equation}
a \frac{du}{dx} = cos(x)
\end{equation}

the exact solution is
\begin{equation}
u = \frac{sin(x)}{a}
\end{equation}

Dirichlet boundary condition is applied on both end.

For the standard continuous Galerkin FEM, we shall first apply the basis function $ \phi_{i} $ on each element. The solution becomes as
\begin{equation}
u(x) = \sum_{i = 1}^{N} \phi_{i} (x) U_{i}
\end{equation}

We integrate the governing equation and obtain the weak form
\begin{equation}
\int_{\Omega} a \frac{du}{dx} \phi_{i} dx = \int_{\Omega} cos(x) \phi_i dx
\end{equation}

Then we apply integration by parts

\begin{equation}
-\int_{\Omega} a u \phi_{i, x} dx + [a u \phi_{i}]^{right}_{left} = \int_{\Omega} cos(x) \phi_{i} dx
\end{equation}

we can substitute the continuous unknown variable equation by 
\begin{equation}
u(x) = \sum_{i = 1}^{N} \phi_{i}(x) U_{i}
\end{equation}

So we obtain the new governing equation in matrix form
\begin{equation}
-\sum_{i = 1}^{N} U_{i} a \int_{\Omega} \phi_{i}\phi_{j,x} dx + BC = \int_{\Omega} cos(x) \phi_{j} dx
\end{equation}

the linear system is simplified as
\begin{equation}
\mathbf{A} \mathbf{x} = \mathbf{b}
\end{equation}

The main distinct feature of DG-FEM is that there is no continuity constraint between element. In another word, the basis functions along the computational domain is discontinuous.

If we apply the original governing equation that
\begin{equation}
a \int_{\Omega} \frac{du}{dx} \phi_{i}  dx = \int_{\Omega} cos(x) \phi_{i} dx
\end{equation}

after the discretization, we have the assemble form that
\begin{equation}
\sum_{j} \int_{j} a \frac{du}{dx} \phi_{i} dx = \sum_{j} \int_{j} cos(x) \phi_{i} dx
\end{equation}

then we apply the integration by parts
\begin{equation}
\sum_{j} \{ -\int_{j} a u \phi_{i, x} dx + [(a u)\phi_{i}]^{j + \frac{1}{2}}_{j - \frac{1}{2}} \} = \sum_{j} \int_{j} cos(x) \phi_{i} dx
\end{equation}

for every element, we have the elemental form 
\begin{equation}
-\int_{j} au\phi_{j,x} dx + [(au)\phi_{i}]^{j + \frac{1}{2}}_{j - \frac{1}{2}} = \int_{j} cos(x) \phi_{j} dx
\end{equation}

We can find that a penalty term is applied on the matrix form. We have several methods to calculate it. Upwinding method is a common useful tool to calculate the $ u $ value on the interface.

Comparing to CG, DG method introduces additional DOFs to maintain the continuity. The main advantage of DG method is that we can easily introduce higher order interpolation to obtain higher order accuracy.

There are several good iterative schemes such as blocak Jacobi, Conjugate Gradient method are stable for higher order. Meanwhile, it's easy to parallelize the computational domain with domain decomposition scheme.


\subsection{Mixed Finite Element Method}

The mixed finite element method is a new type of FEM in which an extra independent variables are introduced which discretization of a partial differential equation. In another word, it is characterized by a variational equation. Usually, the extra variables are constrained by an external vector, Lagrange multiplier. The major difference between FEM and mixed FEM is that the extra independent variables. The mixed finite element method has an advantage on computing the elasticity equation and the stress and strain field. 

To illustrate the mixed FEM, a simple elliptic equation is
\begin{equation}
- \nabla \cdot (a \nabla u) = f \quad in \quad \Omega
\end{equation}

\begin{equation}
u = 0 \quad on \quad \partial \Omega
\end{equation}

we shall introduce an vector $ \mathbf{G} $ that

\begin{equation}
\mathbf{G} = -a \nabla u
\end{equation}

the elliptic problem then can be decomposed into a first order linear system that
\begin{equation}
\mathbf{G} + a \nabla u = 0 \quad in \quad \Omega
\end{equation}

\begin{equation}
\nabla \cdot \mathbf{G} = f \quad in  \quad \Omega 
\end{equation}

\begin{equation}
u = 0 \quad  on  \quad \partial \Omega 
\end{equation}

we can rewrite the first equation as
\begin{equation}
a^{-1} \mathbf{G} + \nabla u = 0 \quad in \quad \Omega 
\end{equation}

then we apply the integration by parts
\begin{equation}
\int_{\Omega} a^{-1} \mathbf{G} \cdot \mathbf{v} d\Omega - \int_{\Omega} c \nabla \cdot \mathbf{v} d\Omega = 0 \quad \mathbf{v} \in H(div, \Omega)
\end{equation}

\begin{equation}
\int_{\Omega} \psi \nabla \cdot \mathbf{G} d\Omega = \int_{\Omega} f \psi d\Omega 
\end{equation}

where 
\begin{equation}
H(div, \Omega) = \{\mathbf{v} \in L^{2}(\Omega)^{d} : \nabla \cdot \mathbf{v} \in L^{2}(\Omega) \}
\end{equation}

We use the Raviart-Thomas spaces to define a generic element $ T \in \Omega $
\begin{equation}
RT_{k} = (P_{k})^{d} + xP_{k}
\end{equation}
where $ k  $ is the order of polynomials degrees.

the variational function $ \mathbf{G} $ and $ u $ can be approximated by 
\begin{equation}
\tilde{G} = \sum_{i = 1}^{N}g_{i} \mathbf{v}_{i}
\end{equation}

\begin{equation}
\mathbf{u} = \sum_{i = 1}^{N} u_{i} \psi_{i}
\end{equation}
both $ \mathbf{v} $ and $ \phi_{i} $ are vector basis functions. If we consider the $ RT_0 $ space, 
\begin{equation}
\mathbf{v} = \begin{pmatrix}
ax + b \\ ay + c\\
\end{pmatrix}
\end{equation}

for each element $ T_{i} , (i = 1, 2, 3)$
\begin{equation}
v_{i} = \begin{pmatrix}
a_{i} x + b_{i} \\ a_{i} y + c_{i}\\
\end{pmatrix}
\end{equation}

the above equation follows the Kronecker property \cite{davio1981kronecker}

\begin{eqnarray}
&&\int_{T} v_{i} \cdot \mathbf{n} dS = \int_{T} (ax + b)n_{x} + (ay + c)n_{y} dS\\
&=& a\int_{x_1}^{x_2} (x_1 n_x + y_1 n_y) \frac{1}{n_y} dx + \int_{x_1}^{x_2}(bn_x + cn_y) \frac{1}{n_y} dx  \\
&=& a(x_1 n_x + y_1 n_y) \frac{x_1 - x_2}{n_y} + (bn_x + cn_y) \frac{x_1 - x_2}{n_y} \\
&=& \delta_{ij}
\end{eqnarray}

Then we apply the divergence theorem, we have
\begin{eqnarray}
\int_{T} \nabla \cdot \mathbf{v} dx dy & = & \int_{T} (\frac{\partial \mathbf{v}}{\partial x} + \frac{\partial \mathbf{v}}{\partial y}) dS\\
&=& \int_{T} 2a dS \\
&=& 2a |T_{i}|
\end{eqnarray}


%---------------------------
\section{Weak Galerkin Finite Element Methods}

\subsection{Preliminary}
In this section, all computation is in Sobolev space\cite{brenner2007mathematical, ciarlet2002finite}. The Lipschitz boundary is in open area $ D \subset R^{d} $, $ d = 2, 3 $. The inner product is defined as
\begin{equation}
|v|_{s, D} = \begin{pmatrix}
\sum_{|\alpha| = s} \int_{D} |\partial^{2} v|^{2} dD
\end{pmatrix}^{1/2}
\end{equation}

where
\begin{equation}
\alpha = (\alpha_{1}, \dots, \alpha_{d})
\end{equation}
\begin{equation}
|\alpha| = \alpha_{1} + \cdots + \alpha_d
\end{equation}

The definition of divergence is defined by
\begin{equation}
H(div; D) = \{ \mathbf{v} : \mathbf{v} \in [L^{2}(D)]^{d}, \nabla \cdot \mathbf{v} \in L^{2}(D) \}
\end{equation}
the norm is defined as
\begin{equation}
||\mathbf{v}||_{H(div, D)} = (||\mathbf{v}||^{2}_{D} + ||\nabla \cdot \mathbf{v}||^{2}_{D})^{1/2}
\end{equation}

the curl of $ H(curl; D) $ in $ L^{2}(D) $ is defined as
\begin{equation}
H(curl; D) = \{\mathbf{v} : \mathbf{v} \in [L^{2}(D)]^{d}, \nabla \times \mathbf{v} \in L^{2}(D)\}
\end{equation}
the norm of it is defined as
\begin{equation}
||\mathbf{v}||_{H(curl; D)} = (||\mathbf{v}||^{2}_{D} + ||\nabla \times \mathbf{v}||^{2}_{D})^{1/2}
\end{equation}

\subsection{Weak operators}

Let's assume that $ T \subset R^{d} $ is an arbitrary polygon domain, the boundary is $ \partial T $. The weak function in the domain is $ v = \{v_{0}, v_{b}\} $, so that $ v_{0} \in L^{2}(T) $ and $ v_{b} \in L^{2} (\partial T) $. The $ v_{0} $ represents the unknown variable vector belongs to the interior domain, and $ v_{b} $ represents the unknown variable vector along the boundary of $ T $. The keynote is that the $ v_{0}  $ is independent with $ v_{b} $ on $ \partial T $. The weak space for all $ T $ is $ S(T) $

\begin{equation}
S(T) = \{ v = \{v_{0}, v_{b}\} : v_{0} \in L^{2}(T), v_{b} \in L^{2} (\partial T) \}
\end{equation}

Now we define some common used weak operators

The weak gradient operator, for any $ v \in S(T) $, the weak gradient of $ v $ is $ \nabla_{w} v $
\begin{equation}
\langle \nabla_{w} v, \mathbf{q} \rangle_{T} = -(v_{0}, \nabla \cdot \mathbf{q})_{T} + \langle v_{b}, \mathbf{q} \cdot \mathbf{n} \rangle_{\partial T}
\end{equation}

the discrete weak gradient operator is $ \nabla_{w, r} v $
\begin{equation}
(\nabla_{w,r} v, \mathbf{q})_{T} = -(v_{0}, \nabla \cdot \mathbf{q})_{T} + \langle v_{b}, \mathbf{q} \cdot \mathbf{n} \rangle_{\partial T}
\end{equation}
where
\begin{equation}
V(T) = \{ \mathbf{v} = \{  \mathbf{v}_{0}, \mathbf{v}_{b} \} : \mathbf{v}_{0} \in [L^{2} (T)]^{d}, \mathbf{v}_{b} \in [L^{2} (\partial T)]^{d} \}
\end{equation}

The weak divergence operator, for any $ v \in S(T) $, is $ \nabla \cdot \mathbf{v} $
\begin{equation}
\langle \nabla_{w} \cdot \mathbf{v}, \varphi \rangle_{T} = -(\mathbf{v}_{0}, \nabla \varphi)_{K} + \langle \mathbf{v}_{b} \cdot \mathbf{n}, \varphi \rangle_{\partial T}
\end{equation}

the discrete weak divergence operator is 
\begin{equation}
(\nabla_{w, r, T} \cdot \mathbf{v}, \varphi)_{T} = -(\mathbf{v}_{0}, \nabla \varphi)_{T} + \langle \mathbf{v}_{b} \cdot \mathbf{n}, \varphi \rangle_{\partial T}
\end{equation}

the curl operator , for any $ v \in S(T) $, is $ \nabla_{w} \times \mathbf{v}$ which is defined as 
\begin{equation}
\langle \nabla_{w} \times \mathbf{v}, \varphi \rangle_{T} = (\mathbf{v}_{0}, \nabla \times \varphi)_{T} - \langle \mathbf{v}_{b} \times \mathbf{n}, \varphi \rangle_{\partial T}
\end{equation}

the discrete weak divergence operator is 
\begin{equation}
(\nabla_{w, r, T} \times \mathbf{v}, \varphi)_{T} = (\mathbf{v}_{0}, \nabla \times \varphi)_{T} - \langle \mathbf{v}_{b} \times \mathbf{n}, \varphi \rangle_{\partial K}
\end{equation}

\subsection{Second order elliptic equation in WG}
In this section, we use the weak operators to solve second order elliptic equation. In domain $ \Omega $, we have the equation in form
\begin{equation}
- \nabla \cdot (a \nabla u) = f
\end{equation}

Considering the Dirichlet boundary condition, we have
\begin{equation}
u = -g, \quad on \quad \partial \Omega
\end{equation}

For Neumann boundary condition, we have
\begin{equation}
(a \nabla u) \cdot \mathbf{n} = -g, \quad on \quad \partial \Omega
\end{equation}

The primal formulation for Dirichlet boundary condition is 
\begin{equation}
(a \nabla u, \nabla v) = (f, v)
\end{equation}

for Neumann boundary condition
\begin{equation}
(a \nabla u, \nabla v) = (f, v) - \langle g, v \rangle_{\partial \Omega}
\end{equation}

The Primal-Mixed formulation is to make $ \mathbf{q} = -a \nabla u $, the elliptic equation is 
\begin{equation}
a^{-1}\mathbf{q} + \nabla u = 0,
\end{equation}

\begin{equation}
\nabla \cdot \mathbf{q} = f.
\end{equation}

we choose assistant function $ \mathbf{p} \in [L^{2}(\Omega)]^{d} $, the bilinear form is
\begin{equation}
(a^{-1} \mathbf{q}, \mathbf{p}) + (\nabla u, \mathbf{p}) = 0;
\end{equation}

for any $ v \in H_{0}^{1} (\Omega) $
\begin{equation}
(\mathbf{q}, \nabla v) = -(f, v)
\end{equation}

For Dirichlet boundary condition, $ u \in H^{1}(\Omega) $, $ \mathbf{q} \in [L^{2} (\Omega)]^{d} $, so that we enforce the boundary condition $ u = -g  $ on $ \partial \Omega $
\begin{equation}
(a^{-1}, \mathbf{q}, \mathbf{p}) + (\nabla u, \mathbf{p})
\end{equation}

\begin{equation}
(\mathbf{q}, \nabla v) = -(f, v),
\end{equation}

for Neumann boundary condition
\begin{equation}
(a^{-1} \mathbf{q}, \mathbf{p}) + (\nabla u, \mathbf{p}) = 0,
\end{equation}

\begin{equation}
(\mathbf{q}, \nabla v) = \langle g, v \rangle_{\partial \Omega} - (f, v),
\end{equation}

Let's define the weak function on each element space
\begin{equation}
S(k, T) = \{ v = \{ v_{0}, v_{b} \} : v_{0} \in P_{k} (T), v_{b}|_{e} \in P_{k} \}
\end{equation}
where $ k $ is the order of polynomial for each function, and $ e $ is the boundary edge of each element.

The weak space $ S_{h} $ is defined as
\begin{equation}
S_{h} = \{ v = \{ v_{0}, v_{b} \} : v |_{T} \in S(k, T), v_{b} |_{\partial T} \}
\end{equation}

for each element $ T $, we use $ Q_{0} $ to represent the $ L^{2} $ projection from $ L^{T} $ to $ P_{k}(T) $ and $ Q_{b} $ is the $ L^{2} $ projection from $ L^{2}(T) $ to $ P_{k}(\partial T) $. The local weak discrete space $ Q_{h} $ is
\begin{equation}
Q_{h} v = \{ Q_{0} v_{0}, Q_{b}, v_{b} \}
\end{equation}

in space $ S_{h} $ we have the bilinear form
\begin{equation}
a(u, v) = \sum (a \nabla_{w} u, \nabla_{w} v)_{T}
\end{equation}

To ensure the continuity between each WG element, a stabilizer is introduced. The stabilizer describes the difference of boundary values and the interior value projected on boundary. Since the interior space and boundary space is independent, the result values are different consequently. However, if the difference is beyond certain threshold, the result value is not continuous across the boundary which leads to break of bilinear form. To solve this problem and give maximum freedom to order of polynomial selection, we use a stabilizer to control the difference and lock to solution in each element. So that the interior value and boundary value are hinged through stabilizer, which has the form,

\begin{equation}
s(u, v) = \rho \sum h_{T}^{-1} \langle Q_{b} u_{0} - u_{b}, Q_{b} v_{0} - v_{b} \rangle_{\partial T}
\end{equation}
where $ \rho $ is commonly set as $ 1 $, and $ h $ is the characteristic length of each element. The governing equation is 

\begin{equation}
a_{s} (u, v) = a(u, v) + s(u, v)
\end{equation}

\subsection{Weak Galerkin triangular meshes}

Consider triangular element linear type basis function for both interior and boundary subspaces $ P_{1}(T) / P_{1} (\partial T) $
\begin{equation}
\phi_{k} = \{ \lambda_{k}, 0 \}, \qquad k = 1,2, 3
\end{equation}
\begin{equation}
\phi_{3 + l} = \{ 0, \mu_{l} \}, \qquad l = 1, 2, \cdots , 2N
\end{equation}
where N is the number of element boundaries.

\begin{figure}[h]
	\centering
	\begin{tabular}{c}
		\includegraphics[width=0.5\textwidth]{./pics/triangle.pdf}
	\end{tabular}
	\caption{\footnotesize Weak Galerkin triangular elements and solution points.}\label{fig1: triangle}
\end{figure}

In this section, we present a triangular WG element with linear interior and boundary space. There are three interior basis functions represent the $ x, y $ and $ xy $ respectively. Meanwhile, there are six boundary basis functions represents six solution points based on the location of Gaussian points.

\subsection{Weak Galerkin quadrilateral meshes}

Consider triangular element linear type basis function for both interior and boundary subspaces $ Q_{1}(T) / Q_{1} (\partial T) $
\begin{equation}
\phi_{k} = \{ \lambda_{k}, 0 \}, \qquad k = 1,2, 3
\end{equation}
\begin{equation}
\phi_{3 + l} = \{ 0, \mu_{l} \}, \qquad l = 1, 2, \cdots , 2N
\end{equation}
where N is the number of element boundaries.
\begin{figure}[h]
	\centering
	\begin{tabular}{c}
		\includegraphics[width=0.8\textwidth]{./pics/quad.pdf}
	\end{tabular}
	\caption{\footnotesize Weak Galerkin quadrilateral elements and solution points.}\label{fig2: quad}
\end{figure}

In this section, we present a quadrilateral WG element with linear interior and boundary space. There are three interior basis functions represent the $ x, y $ and $ xy $ respectively. Meanwhile, there are eight boundary basis functions represents six solution points based on the location of Gaussian points.

%---------------------------------------------------------------------------
