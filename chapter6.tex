\chapter{Conclusions and Future Work}

\section{Conclusion}

In this dissertation, we have successfully designed a novel parallel computing method for solving linear elasticity problems. This method integrates the newly developed weak Galerkin (WG) finite element method with a balancing domain decomposition with constraints (BDDC). The WG-BDDC method is implemented by using Fortran and MPI libraries. The WG-BDDC method is demonstrated to have optimal order of accuracy and convergence properties for both 2nd- and 3rd- order numerical discretization. This method is also proven to have outstanding scalability with superlinear speedup when the number of processors increases by testing up to 600 processors. The condition numbers of the Lanczos matrix from the global primal problem for all test cases presented in this paper are well bounded demonstrating fast and robust convergence properties.

%In hemodynamic perspective, we designed a study on peripheral arterial disease. Based on the collected data, we proved our following predictions: the post-stenotic flow becomes highly unsteady when the narrow degree is over $ 60\% $ area deduction. More than 3 successive $ 50\% $ stenoses cause more streamwise pressure drop than one single $ 60\% $ stenosis. Moreover, pressure drop introduced by  subcritical stenoses grows linearly as the increasing number of applied stenoses. In conclusion, the cross-sectional area reduction is the main factor which contributes to the total.

\section{Future Work}

The Future Work consists of two major steps: 1) further develop the WG-BDDC scheme for nonlinear elasticity problems, 2) extend the current WG-BDDC to tetrahedron and hexahedron elements and incorporate the IMEX coupling scheme to implement an efficient fluid-structural interaction solver.

The present WG-BDDC method is built on triangular and quadrilateral 2 dimensional elements for solving linear elasticity problems. To solve more complicated real-world problems, the nonlinear elasticity equation and three dimensional geometry model are urgent. Our ultimate goal is to simulate complex real-world engineering problems with our efficient high-order accuracy structural and fluid solver. The IMEX coupling scheme will combine the two independent solvers together and are expected to produce high-fidelity results.