\chapter{Conclusions and Future Work}

\section{Conclusions}

In this dissertation, we successfully implement the weak Galerkin finite element method for solving elasticity equation on parallel computers.  We obtained optimal convergence rate, expected order of accuracy and excellent scalability from our WG solvers. The results showed solid performance and great capability of the WG method for solving large scale structural problems on supercomputer cluster. The WG-BDDC solver is an optimal candidate to construct a robust and efficient partitioned parallel fluid-structural interaction solver.

Firstly, we developed a novel hybrid element combining the weak Galerkin finite element with the classic continuous Galerkin finite element. It is designed by inserting multiple CG elements into one WG elements and considered the hybrid element as a non-overlapping subdomain. This hybrid element owns the advantages of the discontinuous feature from the WG method and the computational convenience from the CG method. The hybrid WG-CG element is implemented by using Schur complement method and MPI library for parallel computing. The second order of accuracy is illustrated with different number of CG elements in the hybrid WG-CG element for interior and boundary unknown variables. For linear and nonlinear elasticity equation, we compare the results between our parallel hybrid WG-CG element and an established serial CG solver. The results are identical with an average difference down to $ 10^{-6} $ for Dirichlet and Neumann boundary conditions. The hybrid element method achieved superlinear speedup up to 60 processors on supercomputing cluster, ColonialOne, GWU.


We designed an improved stabilizer for implementing implicit time marching scheme for the WG method and the hybrid WG-CG element method for solving the equation of motion. In this improved stabilizer, the acceleration vector replaces the displacement vector as the unknown variable to calculate the boundary integral. The result matrix completes the information of the mass matrix for the boundary unknown variables. In every time step, the new acceleration can be calculated with the help of the new mass matrix. By using the improved stabilizer, we can compute the implicit time marching scheme with the very low computational cost. 


The hybrid WG-CG element method and improved stabilizer built a solid foundation for developing 2-D parallel WG solver. To overcome the new challenge of increasing interfacial unknown variables, we have successfully designed a novel parallel computing method for solving linear elasticity problems. This method integrates the newly developed weak Galerkin (WG) finite element method with a balancing domain decomposition with constraints (BDDC). The WG-BDDC method is implemented by using Fortran and MPI libraries. The WG-BDDC method is demonstrated to have optimal order of accuracy and convergence properties for both 2nd- and 3rd- order numerical discretization. This method is also proven to have outstanding scalability with superlinear speedup when the number of processors increases by testing up to 600 processors. The condition numbers of the Lanczos matrix from the global primal problem for all test cases presented in this paper are well bounded demonstrating fast and robust convergence properties.

%In hemodynamic perspective, we designed a study on peripheral arterial disease. Based on the collected data, we proved our following predictions: the post-stenotic flow becomes highly unsteady when the narrow degree is over $ 60\% $ area deduction. More than 3 successive $ 50\% $ stenoses cause more streamwise pressure drop than one single $ 60\% $ stenosis. Moreover, pressure drop introduced by  subcritical stenoses grows linearly as the increasing number of applied stenoses. In conclusion, the cross-sectional area reduction is the main factor which contributes to the total.

\section{Future Work}

The Future Work consists of two major steps: 1) further develop the WG-BDDC scheme for nonlinear elasticity problems, 2) extend the current WG-BDDC to handle 3 dimensional geometry and meshes including the tetrahedron and the hexahedron elements.


The present WG-BDDC method is built on triangular and quadrilateral 2 dimensional elements for solving linear elasticity problems. To solve more complicated real-world problems, the nonlinear elasticity equation and three dimensional geometry model are needed. Our ultimate goal is to simulate complex real-world engineering problems with our efficient high-order accuracy structural and fluid solver. The IMEX coupling scheme will combine the two independent solvers together and are expected to produce high-fidelity results.